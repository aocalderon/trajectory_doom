\vspace{-15pt}
\section{Network}
The early 90s predated the wide adoption of the Internet and Wi-Fi. Connecting computers together was difficult and expensive\footnote{Computer-to-computer games existed since the early 1980s. Some, like Battle Chess, were even cross-platform.}. Even if you had the means, bandwidth and latency were abysmal. Most of the time, playing with friends meant getting all your computers into the same room (a LAN party). Playing from the comfort of your room was extremely uncommon. Amusingly, an unconnected computer is now deemed useless. Communication with other machines is something natural and the bare minimum for a machine to be useful.\\
\par
 But back in the early 90s, to pack your 50lb machine (including the CRT) on your bike, make it to a friend's place alive, plug in the cables, start \doom{} and finally see your character move on the other computer's screen was an indescribable feeling. To witness machines actually communicating felt unreal and almost magical.\\
\par 
To achieve the impossible, players had three technologies available: Null-Modem cable, modems, and LAN via network cards.









\subsection{Null-Modem Cable}
The cheapest way and what most people used was the \$20 cable known as a "Null-Modem" which was directly plugged in each PC's COM port. The cable offered no modulation at all (hence the name). For obvious reasons, only two players could participate.\\
\par 
\cscaledimage{0.5}{nullmodemcable.png}{Null-Modem cable}
\par
 A two player game may sound lame by today's standards but back then it was so new and cool that it felt like the most amazing thing in the world.







\subsection{BNC 10Base2 LAN (Local Area Network)}
To play with more than one opponent was substantially more difficult. Besides the relatively easy financial burden of buying the equipment, you had to overcome the much more difficult task of convincing a parent to let four teenagers come to their house where they would scream all night. The famous saying, "fool me once, shame on you; fool me twice, shame on me" is rumored to have originated from betrayed mothers and fathers who had been \textit{doomed} all night.\\
\par



\begin{wrapfigure}[6]{r}{0.25\textwidth}
\centering
\scaledimage{.25}{BNC_T-piece.png}
\end{wrapfigure}

Leaving creative ways to ask for forgiveness aside, on the technical side a player had to plug in a 10Base2 network card via the ISA bus. \\
\par The card had a BNC connector upon which was to be plugged a T-shaped connector known as a T-piece. Each PC node was connected to up to two other nodes via 10Base2 coaxial cables. There was no central point in this type of networking; all machines involved in the network formed a chain. At both ends of the chain a signal terminator had to be connected to prevent an RF signal from being reflected back from each end, causing interference, or power loss.


\begin{wrapfigure}[8]{r}{0.4\textwidth}
\centering
\scaledimage{.4}{BNC_connector_50_ohm_male}
\end{wrapfigure}
The coaxial cables were bulky and so were the connectors. Connecting an end to a T-piece connector was fully achieved with a cool quarter turn of the coupling nut.\\
\par
 % Once physically connected, there was no configuration required (IPX is a network-level protocol like Ethernet). The network card MAC addresses were enough to run the IPX protocol that most games used.\\
 
Once physically connected, games relied on the Internetwork Packet Exchange (IPX) which is a network level protocol like IP. There was no need to configure the host or the network since, contrary to IP, the IPX protocol was able to use the Ethernet MAC address as the machine's IPX address.\\
 \par
\drawing{10base5bncConnector}{10Base2 BNC based network.}
\par
Figure \ref{10base5bncConnector} shows the four elements of a 1994 LAN. \circled{1} the T-piece connector connecting two \circled{2} coaxial cables, forming a link. Each end of the chain must be closed with two load terminators \circled{3}. The network card connects to both the PC via its ISA bus extension slots \circled{4} and the LAN T-base slots.\\
\par
\trivia{Adding a new machine on the network meant either unplugging one of the T-piece connectors or unplugging a chain terminator. In both cases, the central bus was broken and all other machines lost connectivity. Everybody remembers the one friend who was always late to the LAN, forcing everybody to disconnect so he/she could join. The bandwidth was shared meaning the theoretical 10Mb/s was often closer to 5Mb/s. This does not account for friends who wanted to exchange a 30MiB song in \cw{.wav} format (there was no MP3 at the time).}\\
\par
\trivia{Really fancy people could use a 10baseT network which required a "hub" central device resulting in a star-shaped network.}









\subsection{Modem}
The most fortunate players were able to afford the luxury of networking from home. That was very expensive since they not only had to pay for a modem but they also had to pay for every single minute spent online. Before broadband, modems used phone landlines to connect to the Internet provider. This meant nobody could use the telephone while the connection was active. Anybody picking up the phone in the house created enough disturbance to kill the connection.\\
\par
Internet was unattractive since gaming and accessing Bulletin Board Systems was done by calling phone numbers directly. Finding a cool BBS or a gaming partner phone number was an adventure of its own. If one really wanted to read the few HTML pages available, AOL (America OnLine) offered a package of five hours for \$9.95 with each extra hour billed \$3.50 per unit. A user averaging 2h/day was billed $9.95 + 55 * 3.5 = \$202 $ for a month\footnote{Adjusted to inflation: \$352 in 2018.} not to forget a one-time fee of \$399\footnote{Adjusted to inflation: \$696 in 2018.} for a 9600 baud model .\\
\par
\cscaledimage{0.9}{robotic28-8.png}{US Robotics 28.8k baud modem. The top of the line in 1994.}

 While establishing the initial handshake, the modem speaker was kept open. An attuned ear could easily recognize the different phases of V.X bis transaction, speed negotiation, echo canceller disabling, and modulation mode selection, together making the unforgettable melody of a deathmatch in the making.\\
 \par 
\cfullimage{spectrogram2.png}{18 second spectrogram of a V.34 handshake\protect\footnotemark }
\par
  \footnotetext{Source: "The sound of the dialup, pictured" by Oona R\"{a}is\"{a}nen.}
 
 \begin{figure}[H]
\centering  
\begin{tabularx}{\textwidth}{ R{0.1} L{1.9} }
  \toprule
  \textbf{Stage} &  \textbf{Description} \\
  \toprule 
   
   1 & Modem goes off hook.\\
   2 & Telephone exchange sends a dial tone.\\
   3 & DTMF: Model dials 1-(570)-234-0003 a \doom{} player in Pennsylvania, USA.\\
   4 & Answering modem initiates a V.8 bis transaction.\\
   5 & Answering modem asks caller for a list of its capabilities.\\
   6 & Caller responds to V8 bis initiation, agrees to list its capabilities and request to escape from telephony into information transfer mode.\\
   7 & FSK Data @ 300 bps: I'm capable of full V.8. I can transmit ACK. My country is US and I was made by Net2phone Inc.\\
   8 & FSK Data @ 300 bps: Why don't we use V8 then.\\
   9 & Ok, mode acknowledged. Terminating V.8 bis transaction.\\
   \toprule 
   A & Answering modem disables echo suppressors and cancellers in PSTN.\\
   B & FSK Data @ 300 bps: Repeated 6x Here are my modulation modes: V.34, V.32, v.23 duplex ...\\
   C & FSK Data @ 300 bps: Repeated 3x: I can do any of those. \\
   D & Both modems send a wide-spectrum probing signal in both directions to do measurements on the line.\\
   E & Both modems go to scrambled data \\
   \toprule
\end{tabularx}
\caption{}
\end{figure}
\par




Throughout the '90s, bandwidth steadily improved. Upon \doom{}'s release most modems were capable of 14.4 Kbit/s. Those who downloaded the shareware version in December 1993 had to wait 25 minutes to retrieve the 2,166,955 bytes of the ZIP archive.

 \begin{figure}[H]
\centering  
\begin{tabularx}{\textwidth}{ L{0.2} L{0.3} L{0.5}}
  \toprule
  \textbf{Year} & \textbf{Version} & \textbf{Bandwidth} \\
  \toprule 
   
    1990 & V.32 & 9.6 kbit/s \\
    1991 & V.32bis &  14.4 kbit/s \\
    1994 & V.34 & 28.8 kbit/s \\
    1995 & V.34 & 33.6 kbit/s \\
    1996 & V.90 & 56.0/33.6 kbit/s\\
    1999 & V.92 & 56.0/48.0 kbit/s\\
   
   \toprule
\end{tabularx}
\caption{Modem speeds through the 90s.\protect\footnotemark}
\end{figure}
\footnotetext{Bit rate increased at the expense of latency. A 9600 baud modem played \doom{} better than the default configurations on 56kbit modems. Quake needed more bandwidth than \doom{}'s controller replication, so it became a different tradeoff.}

\vspace{-10pt}
\par
On top of the V.XX hardware communication layer, modems were driven using Hayes commands\footnote{A nice abstraction layer, but \doom{} still has a long file with initialization parameters for 49 modems.}. Notice how the command \cw{ATDT} translated to DTMF in the previous spectrogram.\\% in figure \ref{spectrogram2.png} on page \pageref{spectrogram2.png}.\\

\par
{
% \setlength{\belowcaptionskip}{-30pt}
 \begin{figure}[H]

\centering  
\begin{tabularx}{\textwidth}{ L{0.2} L{0.15} L{0.65}}
  \toprule
  \textbf{Modem A} & \textbf{Modem B} & \textbf{Comments} \\
  \toprule 
   
    ATDT15551234 &	&	Modem A issues a dial command: AT-Get the modem's ATtention; D-Dial; T-Touch-Tone; 15551234-Call this number\\
    \toprule 
    RING  & 	& Modem A begins dialing. Modem B's phone-line rings, and the modem reports the fact.\\
      \toprule 
    & ATA	& Modem B issues answer command.\\
    \toprule 
    CONNECT	& CONNECT	& The modems connect, and both modems report "connect"..\\
    abcdef	& abcdef	& When the modems are connected, any characters typed at either side will appear on the other side.\\
    \toprule 
    & +++	& Modem B issues the modem escape command.\\
    \toprule 
     OK &	& The modem acknowledges it.\\
    \toprule 
    & ATH	& Modem B issues a hang up command.\\
    \toprule 
    NO CARRIER &	OK	& Both modems report that the connection has ended. Modem B responds "OK" as the expected result of the command; modem A says NO CARRIER to report that the remote side interrupted the connection.\\
   \toprule
\end{tabularx}
\caption{AT layer dialog between caller and callee.}
\end{figure}
}

\trivia{The fragility of these connections led to humorous ways to end a message. People would finish forum posts with "Hey! Wait! Don't pick up the ph\{\#`\$\{\%\&`+'\%NO CARRIER".}
