The year of 1993 was a magical one, more so than any other. It was the only time we challenged ourselves as a group to create a game that was as good as anything we could have imagined at the time. We didn't challenge ourselves like that before \doom{}, nor after it. It was the right time to shoot for the stars.\\
\par
Incredibly, and perhaps a bit naively, we made a list of the technological wizardry we planned to create, and boldly stated in a press release in January 1993, that \doom{} would be a major source of productivity loss around the world. We truly believed it, and worked hard that year to make it happen. I don't recommend writing a press release at the start of your project, especially one like that.\\
\par
We did so many new things while creating \doom{}. It was our first 3D game to use an engine that broke away from the 2D paradigm we were in from the start of the company, and even stayed in with Wolfenstein 3D and Spear of Destiny, at least for the map layouts. We wanted to use a video camera to scan in our weapons and monsters because we were using real workstations this time around - the mighty NeXTSTEP computers and operating system of Steve Jobs.\\
\par
Making \doom{} was difficult. We were creating a darker-themed game with our creative director Tom Hall who is an absolutely positive guy, and it was anathema to his design ethos. He laid the initial design groundwork by creating the \doom{} Bible which outlined several design concepts we never implemented, some of which were included in 2016's reboot.\\
\par
The engine was revolutionary in that it represented a type of world that no one had seen on a computer screen before. Angled walls and halls that darken in the distance. A high-framerate nightmare some would call it, but it was a high octane blastfest that opened everyone's eyes to the potential of the PC's gaming future. Today's first-person shooters trace their lineage back to this game that bears the distilled essence of what a shooter should be: balanced weapons, insidious level design, a complementary enemy menagerie, and lots of fast action.\\
\par
Throughout the year we tweaked, and added, and removed elements of the game to make it just right. Gone were the score and lives, remnants of the arcades we grew up in. The items that supported a score were removed. The game was far better for it, and those choices influenced our future designs.\\
\par
The application of Bruce Naylor's binary space partition was a huge advance for 3D rendering speed, and the abstract level design style broke games out of the 90-degree maze wall design rut they had been in for 20 years. This was something new, with textured floors and ceilings, stairs, platforms, doors, and blinking lights. We loved having this design palette to work with, and it fit well with the subject matter we based the game upon: Hell.\\
\par
As a group, we played Dungeons \& Dragons for years. Our main campaign was destroyed by demons teleporting onto the material plane and destroying everything in it. This gave us the idea of a demonic invasion, but we decided to base it in the future where we could have some really powerful weapons. Besides, the combination of Hell and science fiction was too great to ignore. We felt even the storyline was slightly new because of it.\\
\par
Writing the DoomEd map editor to create levels was a dream. I was finally using a real operating system with an incredible programming language, Objective-C, and getting to program in a way I had never known. The fact that we had monitors at 1120x832 let us see our game in a way we couldn't under DOS. Using these tools of the future helped us immensely.\\
\par
There was so much we did that was new, it was a little mind-boggling. We were using high-end workstations, a brand-new 3D engine that allowed for incredible graphics and design expression, graphical scanning of our game sprites, and for the first time we were putting multiplayer into our game with a mode I called Deathmatch because that name just made sense.\\
\par
The inclusion of multiplayer co-op and deathmatch modes changed everything about games. We knew that playing a game as fast and over-the-top as \doom{} would signal a new era. I visualized what E1M7 would look like with two players shooting rockets at each other over a large room and it got me more excited than I had been since Wolfenstein 3D's chaingun audio.\\
\par
We couldn't wait to see what players would do with our game, so we made sure it was open and available to modify all the data we had. We had hoped people would change textures, sounds, and make lots of new levels. We were enabling players to let us play their creations finally. It was a major move that would eventually end up with us releasing the source code. Open your game and your fans will own it, and keep it alive after you're gone.\\
\par
For our small team, we took these huge changes in stride and tried to use them to the edge of their capabilities. The technical stretches we made matched the design stretches we were exploring. I felt that we hit a lot of walls and climbed right over them. When Tom Hall left in August 1993, we quickly hired Sandy Petersen to help us in the final stretch. Dave Taylor came aboard to help us fill out the game.\\
\par
At six developers, we were a tight team. Adrian and Kevin held down the art side confidently, while John Carmack handled the meat of the code. I loved being able to play with all of their output, and added a lot of my own code into the game's environments to support my level designs and Sandy's. When we were finished, we knew that we made something pretty great. We couldn't wait for everyone else to see it.\\
\par
It's been an amazing 25 years, and I must first and foremost thank the fans that made it possible and kept it alive all this time as well as the game press who have always supported \doom{} through its many iterations. Your appreciation of our work means everything. I also must thank John, Adrian, Tom, Sandy, Dave and Kevin. It was our crazy dream that made \doom{} possible. Lastly, I want to thank the current \doom{} team for their great work on the latest \doom{} (I'm not at all involved in it, except as a player). Like everyone else, I am super excited for \doom{} Eternal.\\
\par
Then, here's to a quarter century of Rip and Tear!\\
\par
Cheers,\\
\par
-- John Romero\\
\par
\thispagestyle{plain} % Trick to remove header
