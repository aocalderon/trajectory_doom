In May of 1992, id Software was the rising star of the PC gaming industry. Wolfenstein 3D had established the First Person Shooter genre and sales of the sequel "Spear of Destiny" were skyrocketing\footnote{By the end of 1993, combined sales of Wolfenstein 3D and Spear of Destiny reached over 200,000 units. By the end of 1994, that figure increased to 300,000 units.}. The game engine and the associated tools which had taken years to develop were far above the competition. They had an efficient game production pipeline and the talent to use it well with gorgeous levels and assets. Nobody even came close to challenging them...but for how long? They could have kept milking their technology but the evolution of hardware would have \textit{doomed} them.\\
\par

\par
\fq{Because of the nature of Moore's law, anything that an extremely clever graphics programmer can do at one point can be replicated by a merely competent programmer some number of years later.}{John Carmack}. \\
\par
Competitors were coming with their own games. Since its inception around the technological breakthrough named "Adaptive Tile Refresh", id Software's core value had been innovation. They had already released a sequel to Wolfenstein 3D and it was time to move on. The Right Thing to Do (and the most risky\footnote{Things You Should Never Do (Netscape 6 development), by Joel Spolsky.}) was to throw away everything they had worked so hard to build and start their next game from a blank sheet. Assets, levels, tools, and game engine -- everything would be new and innovative.\\
\par
Before getting started, id Software had to decide what hardware they would target, and then what tools to use. A summary assessment of the consumer landscape showed that PCs had significantly evolved since their last hit:
\begin{itemize}
\item Intel's latest CPU, the 486 announced in 1989, was finally becoming affordable. Providing twice the processing power of the previous generation, more and more customers now declined to go for an "old", twice as slow, Intel 386. 
\item The advent of Microsoft Windows 3.1 and its hungry GUI had prompted hardware manufacturers to offer more powerful graphics adapters. Rendering still had to be done in software but chipsets were faster and had more capacity.
\item Frustrated with the bus bottleneck, vendors had teamed up to produce a new standard. PCs often came equipped with a bus ten times faster than the old legacy ISA, called VESA Local Bus (VLB/VL-Bus). 
\item The price of RAM was dropping significantly. The once-standard 2 MiB of RAM was now forecast to be 4 MiB.
\item The audio ecosystem had become even more fragmented with many SoundBlaster "compatible" audio card clones on the market and also new innovative technology such as the Gravis Ultrasound's wavetable synthesis.\\
\end{itemize}
 \par 
 Not only had the hardware evolved, the software was also different. Better compilers such as Watcom allowed faster code to be generated. There was less need for time-consuming hand-crafted assembly, which was slowly becoming a thing of the past\footnote{Intel would bring that trend back with its super-scalar processor, the Pentium, and make Quake development assembly intensive, but this is another story altogether.}. DOS extenders broke the machine free from 16-bit programming and its infamous limited 1 MiB address space.\\
 \par
 On the developer hardware side, new options had appeared. Powerful workstations were now available and affordable to professionals. One company in particular, founded by Steve Jobs after his departure from Apple, combined strong hardware with efficient development tools. \NeXT produced impressive machines running on their UNIX-based OS called NeXTSTEP.\\% NextStep were overlooked yet solid productivity boosters.\\
 \par
 In this whirlwind of novelties, it would have been easy to go in the wrong direction. Yet, id Software seems to have made all the right choices. How did they manage to start from nothing and make one of the best games of all time in just eleven months? This is the question this book will attempt to answer.\\
 \par
 To do so, the two first chapters take a close look at the hardware of the time -- not only the IBM PC on which \doom{} ran but also the \NeXT machines which id Software elected as the foundation of its production pipeline. The third chapter focuses on the team and the tools they wrote to bridge the hardware and the software. With all these capabilities and restrictions in mind the last chapters are a deep dive into the game engine which hopefully will help the reader to appreciate why things are designed the way they are.\\
\par
Now load your shotgun, pack a few medkits and let's dive!\\
\par
% \vspace{2cm}
\par
{
\setlength{\abovecaptionskip}{15pt}
\cfullimage{demonandshotguns2.png}{"\doom{} means two things: demons and shotguns!" -- John Carmack}
}
